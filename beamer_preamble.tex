\documentclass[11pt]{beamer}
\usepackage[utf8]{inputenc}

\usetheme{Boadilla}
\usecolortheme{dolphin}

\usepackage{color,soul}
\usepackage{multicol}
\usepackage{multirow}
\usepackage{amsmath}
\usepackage{amsfonts}
\usepackage{amssymb}
\usepackage{algpseudocode} % uses algorithmicx package automatically
\usepackage{mathrsfs}
\usepackage{graphicx}
\usepackage{tikz}
\usepackage{etex}
\usetikzlibrary{calc}
\usetikzlibrary{shapes.geometric}
\usepackage{pgfplots}
\usepackage{subcaption}
\usepackage{tikz-qtree}
\usepackage{tikz-dependency}
\tikzstyle{vertex}=[draw,fill=black!15,circle,minimum size=10pt,inner sep=1pt]


\DeclareMathOperator*{\argmin}{{arg\,min}}
\DeclareMathOperator*{\argmax}{{arg\,max}}
\DeclareMathOperator{\F}{\mathcal{F}} % Function Classes
\DeclareMathOperator{\FF}{\mathcal{F}} % prettier function classes
\DeclareMathOperator{\Hs}{\mathscr{H}} % Hilbert Spaces
\DeclareMathOperator{\R}{\mathbb{R}} % Reals
\DeclareMathOperator{\Rad}{\mathcal{R}} % Rademacher
\DeclareMathOperator{\E}{\mathbb{E}} % Expectation
\DeclareMathOperator{\Or}{\mathcal{O}} % Order Notation
\DeclareMathOperator{\Tr}{\textbf{Tr}} % Expectation
\DeclareMathOperator{\grad}{\nabla} % Gradient
\DeclareMathOperator{\LLH}{\mathcal{L}} % Log Likelihood etc.
\DeclareMathOperator{\Lag}{\mathcal{L}} % Lagrangian etc.
\DeclareMathOperator{\X}{\mathcal{X}} % input space X
\DeclareMathOperator{\Y}{\mathcal{Y}} % output space Y
\DeclareMathOperator{\bF}{\mathbf{F}} 
\DeclareMathOperator{\w}{\mathbf{w}} % weight vector
\DeclareMathOperator{\y}{\mathbf{y}} % output structure
\DeclareMathOperator{\x}{\mathbf{x}} % input structure
\DeclareMathOperator{\f}{\mathbf{f}} % function
\DeclareMathOperator{\K}{\mathcal{K}} % set of Kernels
\DeclareMathOperator{\vw}{\overrightarrow{w}} % vector w
\DeclareMathOperator{\T}{\mathcal{T}} % # of training eg.

\newcommand{\highlightchunk}[2]{{ \color{#1} [ } #2 { \color{#1} ] }}

\newcommand{\opt}[1]{{#1}^{*}} % give f* for Optimal, Dual ...
\newcommand{\pred}[1]{\hat{#1}} % Prediction 
\newcommand{\dotprod}[2]{ \langle {#1} , {#2} \rangle } % <w,x> style
\renewcommand{\Pr}{\mathbb{P}} % Probability
\renewcommand{\vec}[1]{\mathbf{#1}} % vectors

\newcommand*{\Let}[2]{\State {#1} $\gets$ {#2}}
\newcommand{\Forall}{\,\forall\,} % for all with spacing
\usepackage{xcolor}

\newcommand{\highlight}[1]{\colorbox{yellow}{$\displaystyle #1$}}
\newcommand{\highlightmath}[1]{\colorbox{yellow}{\[\displaystyle #1\]}}
\setbeamertemplate{footline}[frame number]

\newcommand{\tikzmark}[1]{\tikz[overlay,remember picture] \node (#1) {};}
\definecolor{lightblue}{rgb}{.90,.95,1}
\sethlcolor{lightblue}
\renewcommand<>{\hl}[1]{\only#2{\beameroriginal{\hl}}{#1}}

% http://tex.stackexchange.com/questions/41683/why-is-it-that-coloring-in-soul-in-beamer-is-not-visible
\makeatletter
\newcommand\SoulColor{%
  \let\set@color\beamerorig@set@color
  \let\reset@color\beamerorig@reset@color}
\makeatother
\SoulColor


\usepackage{appendixnumberbeamer}
\setbeamercolor{alerted text}{fg=blue}
\setbeamercovered{invisible}    % invisible future text instead of wash-out

\newtheorem{property}{Property}
\newtheorem{usage}{Usage}
% tikzmark command, for shading over items
\newcommand{\Cross}{$\mathbin{\tikz [x=1.4ex,y=1.4ex,line width=.2ex, red] \draw (0,0) -- (1,1) (0,1) -- (1,0);}$}%
